% Classe appositamente creata per tesi di Ingegneria Informatica all'università Roma Tre
\documentclass{TesiDiaUniroma3}

\usepackage{rotating}

\usepackage{xcolor}
\usepackage{listings}
\lstset{basicstyle=\ttfamily,
  showstringspaces=false,
  commentstyle=\color{red},
  keywordstyle=\color{blue}
}

\usepackage{tikz,pgfplots}

\usepackage[font=small,labelfont=bf,tableposition=top]{caption}
\DeclareCaptionType[fileext=loc]{chart}[Chart][List of Charts]
\usepackage{float}



% --- INIZIO dati relativi al template TesiDiaUniroma3
% dati obbligatori, necessari al frontespizio
\titolo{An Efficient Design of a Next Generation Sequencing pipeline with Apache Spark}
\autore{Nicholas Tucci}
\matricola{461669}
\relatore{Prof. Riccardo Torlone}
\correlatore{Prof. Paolo Missier} % modifica anche TesiDiaUniroma3.cls se vuoi avere un correlatore
\annoAccademico{2016/2017}

% dati opzionali
\dedica{Questa \`e la dedica} % solo se nel documento si usa il comando \generaDedica
% --- FINE dati relativi al template TesiDiaUniroma3

% --- INIZIO richiamo di pacchetti di utilità. Questi sono un esempio e non sono strettamente necessari al modello per la tesi.
\usepackage[plainpages=false]{hyperref}	% generazione di collegamenti ipertestuali su indice e riferimenti
\usepackage[all]{hypcap} % per far si che i link ipertestuali alle immagini puntino all'inizio delle stesse e non alla didascalia sottostante
\usepackage{amsthm}	% per definizioni e teoremi
\usepackage{amsmath}	% per ``cases'' environment
% --- FINE riachiamo di pacchetti di utilità

\begin{document}
% ----- Pagine di fronespizio, numerate in romano (i,ii,iii,iv...) (obbligatorio)
\frontmatter
\generaFrontespizio
\newenvironment{dedication}
  {\clearpage           % we want a new page
   \thispagestyle{empty}% no header and footer
   \vspace*{\stretch{1}}% some space at the top 
   \itshape             % the text is in italics
   \raggedleft          % flush to the right margin
  }
  {\par % end the paragraph
   \vspace{\stretch{3}} % space at bottom is three times that at the top
   \clearpage           % finish off the page
  }
\begin{dedication}
In a way this project has the goal to detect and annotate genome variants which may be deleterious, such as cancer, heart attacks or alzheimer.\newline
So this project is dedicated to my grandma, who left us during my Master Degree due to alzheimer illness.
\end{dedication}

\newpage
\ringraziamenti{ringraziamenti}	% inserisce i ringraziamenti e li prende in questo caso da ringraziamenti.tex
\newpage
\begin{abstract}
The goal of this project is to implement a Next Generation Sequencing Pipeline with Apache Spark. This is a pipeline which processes human genomes in order to detect and annotate Variants which in literature are known as deleterious. These kind of pipeline are much time and resource consuming, so has been introduced Apache Spark, a Big Data technology which can introduce efficiency in the work load. With Spark there is the possibility to deploy this pipeline over a computer-cluster, so through Docker I delivered the necessary technologies to execute it in cluster-mode on Microsoft Azure VMs.
\end{abstract}
\newpage
\introduzione{introduzione}		% inserisce l'introduzione e la prende in questo caso da introduzione.tex
\generaIndice
\listoffigures
\listoftables
\newpage
\listofcharts

% ----- Pagine di tesi, numerate in arabo (1,2,3,4,...) (obbligatorio)
\mainmatter
% il comando ``capitolo'' ha come parametri:
% 1) il titolo del capitolo
% 2) il nome del file tex (senza estensione) che contiene il capitolo. Tale nome \`e usato anche come label del capitolo
\capitolo{Background}{chapter1-background}
\capitolo{Pipeline}{chapter2-pipeline}
\capitolo{Clustering}{chapter3-clustering}
\capitolo{Datasets}{chapter4-datasets}
\capitolo{Performance Analysis}{chapter5-performance_analysis}

% ----- Parte finale della tesi (obbligatorio)
\backmatter
\conclusioni{conclusioni}

% Bibliografia con BibTeX (obbligatoria)
% Non si deve specificare lo stile della bibliografia
\bibliography{bibliografia} % inserisce la bibliografia e la prende in questo caso da bibliografia.bib

\end{document}
